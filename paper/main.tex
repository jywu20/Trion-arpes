\documentclass[hyperref, a4paper]{article}

\usepackage{textgreek}
\usepackage{geometry}
\usepackage{titling}
\usepackage{titlesec}
\usepackage{footnote}
\usepackage[colorinlistoftodos]{todonotes}
\usepackage{booktabs}
\usepackage{array}
\usepackage{multirow}
\usepackage{amsmath, amssymb, amsthm}
\usepackage{mathtools}
\usepackage{bbm}
\usepackage{graphicx}
\usepackage{subcaption}
\usepackage{physics}
\usepackage{tensor}
\usepackage{siunitx}
\usepackage[version=4]{mhchem}
\usepackage{tikz}
\usepackage{xcolor}
\usepackage{listings}
\usepackage{autobreak}
\usepackage[colorlinks,unicode]{hyperref} % , linkcolor=black, anchorcolor=black, citecolor=black, urlcolor=black, filecolor=black
\usepackage{xurl}
\usepackage[most]{tcolorbox}
\usepackage[backend=bibtex,sorting=none,doi=false,isbn=false,url=false]{biblatex}
\addbibresource{trion.bib}
\addbibresource{arpes.bib}

\usepackage{prettyref}

% Page style; to be removed when this article is placed in another template
\geometry{left=3.18cm,right=3.18cm,top=2.54cm,bottom=2.54cm}
\titlespacing{\paragraph}{0pt}{1pt}{10pt}[20pt]
\setlength{\droptitle}{-5em}


% Math operators
\DeclareMathOperator{\timeorder}{\mathcal{T}}
\DeclareMathOperator{\diag}{diag}
\DeclareMathOperator{\legpoly}{P}
\DeclareMathOperator{\primevalue}{P}
\DeclareMathOperator{\sgn}{sgn}
\newcommand*{\ii}{\mathrm{i}}
\newcommand*{\ee}{\mathrm{e}}
\newcommand*{\const}{\mathrm{const}}
\newcommand*{\suchthat}{\quad \text{s.t.} \quad}
\newcommand*{\argmin}{\arg\min}
\newcommand*{\argmax}{\arg\max}
\newcommand*{\normalorder}[1]{: #1 :}
\newcommand*{\pair}[1]{\langle #1 \rangle}
\newcommand*{\fd}[1]{\mathcal{D} #1}
\DeclareMathOperator{\bigO}{\mathcal{O}}


% Embedded codes
\lstdefinestyle{console}{
    basicstyle=\footnotesize\ttfamily,
    breaklines=true,
    postbreak=\mbox{\textcolor{red}{$\hookrightarrow$}\space}
}

% Reference formatting
\newrefformat{fig}{Fig.~\ref{#1}}
\newrefformat{tbl}{Table~\ref{#1}}

% TiKZ settings
\usetikzlibrary{calc}
\tikzset{every picture/.style={line width=0.3pt}} 

% Displaying chemical formula in bookmarkers

\pdfstringdefDisableCommands{%
  \def\\{}%
  \def\ce#1{<#1>}%
}

\pdfstringdefDisableCommands{%
  \def\texttt#1{<#1>}%
  \def\mathbb#1{#1}%
}
\pdfstringdefDisableCommands{\def\eqref#1{(\ref{#1})}}

\makeatletter
\pdfstringdefDisableCommands{\let\HyPsd@CatcodeWarning\@gobble}
\makeatother



\newcommand{\citetime}[1]{item \ref{#1} in \prettyref{sec:timeline}}
\newcommand{\address}[1]{\href{#1}{\url{#1}}}
\newcommand{\shortcode}[1]{\texttt{#1}}

\DeclareSIUnit{\au}{a.u.}

\lstset{style = console}

% Labels 
\newcommand*{\ke}{\vb{k}_{\text{e}}}
\newcommand*{\khi}[1]{\vb{k}_{\text{h#1}}}
\newcommand*{\kh}{\vb{k}_{\text{h}}}
\newcommand*{\re}{\vb{r}_{\text{e}}}
\newcommand*{\rhi}[1]{\vb{r}_{\text{h#1}}}
\newcommand*{\me}{m_{\text{e}}}
\newcommand*{\mh}{m_{\text{h}}}
\newcommand*{\Eg}{E_{\text{g}}}

\title{Trion ARPES signature}
\author{Jinyuan Wu, someone else, Diana Y. Qiu}

\begin{document}

\maketitle

\section{ARPES theory}

When the probe is much weaker than the pump 
and does not itself influence the electronic structure,
the most generic theory of driven ARPES is based on the  
pumped single-electron lesser Green function, 
to which the pump adds a self-energy correction,
\cite{freericks2009theoretical,schuler2021theory};
the theory reduced to the usual Fermi golden rule 
when the pump itself is weak \cite{sobota2021angle,freericks2021two}.
When the pump pulse is strong at its peak but has faded away at the probing time, 
the main effect of pumping is preparing an initial state, 
and the ARPES intensity can be evaluated from 
pure-state first-order time dependent perturbation theory,
which gives a Fermi golden rule-like final result \cite{rustagi2018photoemission}.
Here we generalize the methodology in \cite{rustagi2018photoemission} 
to all possible exciton states caused by the pump and not just excitons.


\section{Two-band model for trion}

In this work, we generalize the two-band model approach to trions \cite{chang2021variationally}
to the indirect band gap case.
We start with the following example.
\begin{equation}
    H = \frac{(\ke - \vb{w})^2}{2\me} + \Eg 
    + \frac{\khi{1}^2}{2\mh} + \frac{\khi{2}^2}{2\mh} 
    + V(\rhi{1} - \rhi{2}) - V(\re - \rhi{1}) - V(\re - \rhi{2}).
    \label{eq:two-band-w}
\end{equation}
Here $\khi{i}$ is the crystal momentum of the quantum state containing hole $i$ only, 
which is the opposite of the momentum of the corresponding electron state of the hole;
similarly, $\vb{k}^2/2\mh$ is the opposite of the dispersion relation of the valence band top,
and therefore $\mh > 0$.

It can be easily seen that the total momentum of the trion 
\begin{equation}
    \vb{P} = \ke + \khi{1} + \khi{2}
\end{equation}
is conserved, and we are to rewrite the Hamiltonian in terms of 
the total momentum, which should only appear in the total kinetic energy term,
and two internal degrees of freedom.
Following the procedure in \cite{chang2021variationally},
we define 
\begin{equation}
    \vb{r}_1 = \rhi{1} - \re, \quad \vb{r}_2 = \rhi{2} - \re 
\end{equation}
as the internal degrees of freedom.
To separate the Hamiltonian into terms about $\vb{P}$ and terms about the internal degrees of freedom,
we need to find the canonical coordinate corresponding to $\vb{P}$
and the canonical momenta corresponding to $\vb{r}_{1, 2}$.

Regarding the canonical coordinate corresponding to $\vb{P}$, 
intuitively we may choose the center of mass 
\begin{equation}
    \vb{R} = \frac{\me \re + \mh \rhi{1} + \mh \rhi{2}}{M}, \quad M = \me + 2 \mh
\end{equation}
as the canonical coordinate corresponding to $\vb{P}$;
this correctness of this choice can be directly verified by  
writing $\vb{R}, \vb{r}_1, \vb{r}_2$ as functions of $\re, \rhi{1}, \rhi{2}$
and verifying that $\partial_{\vb{R}}$ is the sum of 
$\partial_{\re}$, $\partial_{\rhi{1}}$ and $\partial_{\rhi{2}}$,
and hence $\vb{P} = - \ii \hbar \partial_{\vb{R}}$ 
is indeed the total momentum.

We can define the canonical momenta of $\vb{r}_{1,2}$, 
hereafter referred to as $\vb{k}_{1,2}$, 
as $- \ii \hbar_{\vb{r}_{1, 2}}$;
in this way 
\begin{equation}
    \ke = \frac{\me}{M} \vb{P} - \vb{k}_1 - \vb{k}_2, \quad 
    \khi{1,2} = \vb{k}_{1, 2} + \frac{\mh}{M} \vb{P},
\end{equation}
and the kinetic energy part of the \eqref{eq:two-band-w} now reads 
\begin{equation}
    H_{\text{kin}} = \frac{(\vb{P} - \vb{w})^2}{2 M} + \Eg 
    + \frac{\vb{k}_1^2}{2\mh} + \frac{\vb{k}_2^2}{2\mh}
    - \frac{\vb{w}^2}{2M} + \frac{(\vb{k}_1 + \vb{k}_2 + \vb{w})^2}{2\me}.
\end{equation}
We however want to shift the definitions of $\vb{k}$ 
so that the kinetic part resembles the kinetic energy in a direct band gap system.
By redefining 
\begin{equation}
    \vb{k}_{1,2} = - \ii \hbar \vb{r}_{1, 2} + \frac{\vb{w}}{1 + \me / \mu}, \quad 
    \mu = \frac{\me \mh}{\me + \mh},
\end{equation}
and therefore 
\begin{equation}
    \ke = \frac{\me}{M} \vb{P} + \frac{2 \mh \vb{w}}{M} - \vb{k}_1 - \vb{k}_2, \quad 
    \khi{1,2} = \vb{k}_{1,2} - \frac{\mh \vb{w}}{M} + \frac{\mh}{M} \vb{P},
\end{equation}
the total Hamiltonian now reads
\begin{equation}
    H =  \frac{(\vb{P} - \vb{w})^2}{2 M} + \Eg 
    + \frac{\vb{k}_1^2}{2\mh} + \frac{\vb{k}_2^2}{2\mh} + \frac{\vb{k}_1 \cdot \vb{k}_2}{2\me}
    + V(\vb{r}_1 - \vb{r}_2) - V(\vb{r}_1) - V(\vb{r}_2).
\end{equation}
The redefinition of $\vb{k}_{1, 2}$ does not change the commutation relation 
between $\vb{r}_{1, 2}$ and $\vb{k}_{1, 2}$.
We find that the internal part of the Hamiltonian has the same form 
as the Hamiltonian in \cite{chang2021variationally}. 
This can also be seen as generalization of the indirect band gap exciton two-band model to the trion;
for the former we have \cite{rustagi2018photoemission} 
\begin{equation}
    H = \frac{(\vb{Q} - \vb{w})^2}{2M} + \Eg + \frac{\vb{k}^2}{2\mu} - V(\vb{r}), 
\end{equation}
where 
\begin{equation}
    M = \me + \me, \quad 
    \ke = \frac{\me}{M} \vb{Q} + \vb{k} + \frac{\mh}{M} \vb{w}, \quad 
    \kh = \frac{\mh}{M} \vb{Q} - \vb{k} - \frac{\mh}{M} \vb{w}.
\end{equation}

\section{Variational wave function}

The spin part is in a singlet state, 
and therefore the orbital part of the wave function is symmetric.

It should be noted that since $\vb{k}$ is confined on the two-dimensional place, 
so is $\vb{r}$, 
which is the canonical coordinate corresponding to the crystal momentum $\vb{k}$
instead of the free-space momentum, 
and therefore differs from the free-space position;
the wave function in terms of the free-space position does have $z$-directional distribution,
but the wave function in terms of $\vb{r}$ does not.
The spatial part of the wave function 
therefore is to be solved on the 2D plane.
On the other hand, electromagnetism in a two-dimensional system 
is still inherently three-dimensional, 
and the Coulomb force retains its inverse-square form,
leading to the $\sim 1/r$ form of $V(r)$.

\begin{equation}
    \Psi(\vb{r}_1, \vb{r}_2) = 
    \psi_{\text{1s,$a$}}(\vb{r}_1) 
    \psi_{\text{1s,$b$}}(\vb{r}_2)
    + \psi_{\text{1s,$b$}}(\vb{r}_1) 
    \psi_{\text{1s,$a$}}(\vb{r}_2)
\end{equation}

\begin{equation}
    \braket*{\text{1s,$a$}}{\text{1s,$b$}} = \frac{4ab}{(a+b)^2}.
\end{equation}
\begin{equation}
    \mel*{\text{1s,$a$}}{-\laplacian}{\text{1s,$b$}} = \frac{4}{(a+b)^2}.
\end{equation}

\begin{equation}
    \mel*{\text{1s, $a$}}{\frac{1}{r}}{\text{1s, $b$}} = \frac{4}{a+b}
\end{equation}

Serious problem:
\begin{equation}
    \int \dd[2]{\vb{r}_1} \int \dd[2]{\vb{r}_2} \ee^{- r_1 / a} \frac{1}{\abs*{\vb{r}_1 - \vb{r}_2}} \ee^{- r_2 / b}
\end{equation}
One option might be to numerically solve the interaction part, 
and use the analytic forms elsewhere. 

An immediate consequence of this fact is that
since at the ground state 
\begin{equation}
    \ke = \frac{\me}{M} \vb{P} + \frac{2 \mh}{M} \vb{w}.
\end{equation}

The same procedure can be applied to other configurations of peaks and valleys.

Variational wave function: TODO list 
\begin{enumerate}
    \item We should work in the real space, because the form of $V$ can be complicated.
    \item The kinetic term should be evaluated analytically.
    \item Normalization??? Why the factor is $1 / \sqrt{2}$?
    \item The structure of the three-particle wave function.
\end{enumerate}

\section{ARPES signatures}

When $\vb{k}_1$ is set to zero, we find 
\begin{equation}
    \khi{2} = \frac{\me + \mh}{M} \vb{P} + \frac{\mh}{M} \vb{w} - \ke,
\end{equation}

and when $\vb{k}_1 = \vb{k}_2$, we have 

It can be seen that the trion ARPES signature shows a significantly smaller variance with respect to the trion temperature
than the exciton ARPES signature does.


\printbibliography

\end{document}