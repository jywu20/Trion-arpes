\documentclass[%
 reprint,
superscriptaddress,
%groupedaddress,
%unsortedaddress,
%runinaddress,
%frontmatterverbose, 
%preprint,
%preprintnumbers,
%nofootinbib,
%nobibnotes,
%bibnotes,
 amsmath,amssymb,
 aps,
%pra,
prb,
%rmp,
%prstab,
%prstper,
%floatfix,
]{revtex4-2}

\usepackage{textgreek}
\usepackage{booktabs}
\usepackage{array}
\usepackage{multirow}
\usepackage{amsmath, amssymb, amsthm}
\usepackage{mathtools}
\usepackage{bbm}
\usepackage{graphicx}
\usepackage{physics}
\usepackage{tensor}
\usepackage{siunitx}
\usepackage[version=4]{mhchem}
\usepackage{tikz}
\usepackage{xcolor}
\usepackage{listings}
\usepackage{autobreak}
\usepackage[colorlinks,unicode]{hyperref} % , linkcolor=black, anchorcolor=black, citecolor=black, urlcolor=black, filecolor=black
\usepackage{xurl}
\usepackage[most]{tcolorbox}


\usepackage{prettyref}


% Math operators
\DeclareMathOperator{\timeorder}{\mathcal{T}}
\DeclareMathOperator{\diag}{diag}
\DeclareMathOperator{\legpoly}{P}
\DeclareMathOperator{\primevalue}{P}
\DeclareMathOperator{\sgn}{sgn}
\newcommand*{\ii}{\mathrm{i}}
\newcommand*{\ee}{\mathrm{e}}
\newcommand*{\const}{\mathrm{const}}
\newcommand*{\suchthat}{\quad \text{s.t.} \quad}
\newcommand*{\argmin}{\arg\min}
\newcommand*{\argmax}{\arg\max}
\newcommand*{\normalorder}[1]{: #1 :}
\newcommand*{\pair}[1]{\langle #1 \rangle}
\newcommand*{\fd}[1]{\mathcal{D} #1}
\DeclareMathOperator{\bigO}{\mathcal{O}}


% Embedded codes
\lstdefinestyle{console}{
    basicstyle=\footnotesize\ttfamily,
    breaklines=true,
    postbreak=\mbox{\textcolor{red}{$\hookrightarrow$}\space}
}

% Reference formatting
\newrefformat{fig}{Fig.~\ref{#1}}
\newrefformat{tbl}{Table~\ref{#1}}

% TiKZ settings
\usetikzlibrary{calc}
\tikzset{every picture/.style={line width=0.3pt}} 

% Displaying chemical formula in bookmarkers

\pdfstringdefDisableCommands{%
  \def\\{}%
  \def\ce#1{<#1>}%
}

\pdfstringdefDisableCommands{%
  \def\texttt#1{<#1>}%
  \def\mathbb#1{#1}%
}
\pdfstringdefDisableCommands{\def\eqref#1{(\ref{#1})}}

\makeatletter
\pdfstringdefDisableCommands{\let\HyPsd@CatcodeWarning\@gobble}
\makeatother



\newcommand{\citetime}[1]{item \ref{#1} in \prettyref{sec:timeline}}
\newcommand{\shortcode}[1]{\texttt{#1}}

\DeclareSIUnit{\au}{a.u.}

\lstset{style = console}

% Labels 
\newcommand*{\ke}{\vb{k}_{\text{e}}}
\newcommand*{\kei}[1]{\vb{k}_{\text{e#1}}}
\newcommand*{\khi}[1]{\vb{k}_{\text{h#1}}}
\newcommand*{\kh}{\vb{k}_{\text{h}}}
\newcommand*{\kp}{\vb{k}_\parallel}
\newcommand*{\re}{\vb{r}_{\text{e}}}
\newcommand*{\rh}{\vb{r}_{\text{h}}}
\newcommand*{\rhi}[1]{\vb{r}_{\text{h#1}}}
\newcommand*{\rei}[1]{\vb{r}_{\text{e#1}}}
\newcommand*{\me}{m_{\text{e}}}
\newcommand*{\mh}{m_{\text{h}}}
\newcommand*{\Eg}{E_{\text{g}}}


\begin{document}

\title{Trion ARPES signature}
\author{Jinyuan Wu}
\email{jinyuan.wu@yale.edu}
\affiliation{Department of Mechanical Engineering and Materials Science, Yale University, New Haven, CT 06520}
\author{Diana Y. Qiu}
\email{diana.qiu@yale.edu}
\affiliation{Department of Mechanical Engineering and Materials Science, Yale University, New Haven, CT 06520}

\maketitle

\section{ARPES theory}

In most driven ARPES settings, the probe is much weaker than the pump 
and does not itself influence the electronic structure,
and the generic theory of ARPES can be formulated by 
treating the pump as a non-equilibrium self-energy, 
while the effect of the probe can be captured by the first order perturbation  
and hence the final ARPES intensity is proportional to the electron dipole and 
the pumped single-electron lesser Green function
\cite{freericks2009theoretical,schuler2021theory};
the theory reduced to the usual Fermi golden rule 
when the pump is turned off \cite{sobota2021angle,freericks2021two}.
When the pump pulse is strong at its peak but has faded away at the probing time, 
the main effect of pumping is preparing an initial state, 
and the ARPES intensity can be evaluated from 
pure-state first-order time dependent perturbation theory,
which gives a Fermi golden rule-like expression of the probability of an electron being pumped out of the sample~\cite{rustagi2018photoemission}.
Ref.~\cite{rustagi2018photoemission} is focused on excitons;
here we generalize its methodology to all excitations and have  
\begin{widetext}
    \begin{equation}
        I_{\vb{k}}(\omega, t) \propto \sum_{n_1, n_2} \sum_{n}
        \rho_n (M^{fn_2}_{\vb{k} \vb{k}_\parallel})^* M^{fn_1}_{\vb{k} \vb{k}_\parallel} 
        \int_{t_0}^t \dd{t_1} \int_{t_0}^t \dd{t_2}
        \ee^{-\ii (E_n - \omega ) (t_1 - t_2)}
        \mel{\Psi_n}{c^\dagger_{n_2 \vb{k}_\parallel} 
        U(t_2, t_1) c_{n_1 \vb{k}_\parallel}}{\Psi_n} s(t_1) s(t_2),
        \label{eq:exciton-response}
    \end{equation}
\end{widetext}
where $t$ is the observation time, 
$t_0$ is the starting time of the probe pulse,
$s(t)$ is the envelope function of the probe,
$\omega$ is the output electron energy shifted by the frequency of the probe pulse 
and the work function of the sample, 
$\vb{k}$ is the momentum of the outgoing electron 
and $\vb{k}_\parallel$ is its projection on the two-dimensional plane,  
$n_{1, 2}$ are electron band indices, 
$n = (S, \vb{Q})$ is the index of exciton modes, 
$\rho_n$ is the effective density matrix of excitons, 
$\ket*{\Psi_n}$ is the many-body wave function of the exciton state at $t=t_0$,
and $U(t_2, t_1)$ is the time evolution operator.
The $c^\dagger U c \ee^{- \ii E_n t}$ 
factor comes from the electron lesser Green function; 
the $\ee^{- \ii E_n t}$ factor comes from the time evolution of the excitation states, 
while the time evolution operator $U(t_2, t_1)$ 
instructs the time evolution of the residue of the excitation after one electron is driven out; 
time oscillation from the two factors together gives us the energy conservation relation
\begin{equation}
    \omega = E_n - E_{\text{residue}} , 
\end{equation}
where $E_{\text{residue}}$ refers to the remaining energy of the system after
one electron is driven out from $n_1, \kp$.
For three-dimensional systems, the momenta of $c$ and $c^\dagger$ in the $c^\dagger U c$ factor can be different \cite{freericks2009theoretical},  
but for two-dimensional materials however, since by momentum conservation  
the crystal momentum of the band electron that is driven out by the external field 
can only be the parallel component of $\vb{k}$, and we refer it as $\vb{k}_\parallel$ in \eqref{eq:exciton-response}.
As below we are mostly dealing with the ARPES spectrum of a single mode
In principle, the $\mel{\Psi_n}{c^\dagger_{n_2 \vb{k}_\parallel} U(t_2, t_1) c_{n_1 \vb{k}_\parallel}}{\Psi_n}$ matrix can be non-diagonal with an excited initial state;
frequently, however, because of energy conservation, the signature at given $(\omega, \vb{k})$
can only come from a single band, and in this case $n_1 = n_2$.

\section{Two-band model for trion}

In this work, we generalize the two-band model approach to trions \cite{chang2021variationally}
to the indirect band gap case.
We start with the following example.
\begin{equation}
    \begin{aligned}
        H &= \frac{(\ke - \vb{w})^2}{2\me} + \Eg 
        + \frac{\khi{1}^2}{2\mh} + \frac{\khi{2}^2}{2\mh} \\
        &\quad+ V(\rhi{1} - \rhi{2}) - V(\re - \rhi{1}) - V(\re - \rhi{2}).
    \end{aligned}
    \label{eq:two-band-w}
\end{equation}
Here $\khi{$i$}$ is the crystal momentum of the quantum state containing hole $i$ only, 
which is the opposite of the momentum of the corresponding electron state of the hole;
similarly, $\vb{k}^2/2\mh$ is the opposite of the dispersion relation of the valence band top,
and therefore $\mh > 0$, 
and under this Hamiltonian we treat holes as ordinary particles with positive masses and positive charges and ordinary momentum conservation relations without minus signs. 

The ARPES signature of trion mode $(S, \vb{P})$ comes solely from the electron in the trion.
Therefore, the energy conservation relation hidden in \eqref{eq:exciton-response} means 
the dispersion relation of the ARPES signature should always have the form of 
\begin{equation}
    \omega = E_n - \frac{\khi{1}^2}{2\mh} - \frac{\khi{2}^2}{2\mh}, 
\end{equation}
where $\khi{1, 2}$ are implicit functions of $\kp = \ke$ and $\vb{P}$.
Unlike the case for excitons, however, 
momentum conservation does not give a one-to-one mapping from $\ke$ and $\vb{P}$ to $\khi{1, 2}$; 
we still need one additional constraint to pin down $\khi{1, 2}$, 
which is to be given by the position of the trion wave function maximum when $\vb{P}$ and $\ke$ are given, 
which, in turn, needs us to decompose \eqref{eq:two-band-w} into an external part that solely depends on the total momentum of the trion and an internal part that decides the structure of the trion wave function.

It can be easily seen that the total momentum of the trion 
\begin{equation}
    \vb{P} = \ke + \khi{1} + \khi{2}
\end{equation}
is conserved, and we are to rewrite the Hamiltonian in terms of 
the total momentum, which should only appear in the total kinetic energy term,
and two internal degrees of freedom.
Following the procedure in \cite{chang2021variationally},
we define 
\begin{equation}
    \vb{r}_1 = \rhi{1} - \re, \quad \vb{r}_2 = \rhi{2} - \re 
\end{equation}
as the internal degrees of freedom.
To separate the Hamiltonian into terms about $\vb{P}$ and terms about the internal degrees of freedom,
we need to find the canonical coordinate corresponding to $\vb{P}$
and the canonical momenta corresponding to $\vb{r}_{1, 2}$.

Regarding the canonical coordinate corresponding to $\vb{P}$, 
intuitively we may choose the center of mass 
\begin{equation}
    \vb{R} = \frac{\me \re + \mh \rhi{1} + \mh \rhi{2}}{M}, \quad M = \me + 2 \mh
\end{equation}
as the canonical coordinate corresponding to $\vb{P}$;
this correctness of this choice can be directly verified by  
writing $\vb{R}, \vb{r}_1, \vb{r}_2$ as functions of $\re, \rhi{1}, \rhi{2}$
and verifying that $\partial_{\vb{R}}$ is the sum of 
$\partial_{\re}$, $\partial_{\rhi{1}}$ and $\partial_{\rhi{2}}$,
and hence $\vb{P} = - \ii \hbar \partial_{\vb{R}}$ 
is indeed the total momentum.

We can define the canonical momenta of $\vb{r}_{1,2}$, 
hereafter referred to as $\vb{k}_{1,2}$, 
as $- \ii \hbar_{\vb{r}_{1, 2}}$;
in this way 
\begin{equation}
    \ke = \frac{\me}{M} \vb{P} - \vb{k}_1 - \vb{k}_2, \quad 
    \khi{1,2} = \vb{k}_{1, 2} + \frac{\mh}{M} \vb{P},
\end{equation}
and the kinetic energy part of the \eqref{eq:two-band-w} now reads 
\[
    H_{\text{kin}} = \frac{(\vb{P} - \vb{w})^2}{2 M} + \Eg 
    + \frac{\vb{k}_1^2 + \vb{k}_2^2}{2\mh} 
    - \frac{\vb{w}^2}{2M} + \frac{(\vb{k}_1 + \vb{k}_2 + \vb{w})^2}{2\me}.
\]
We however want to shift the definitions of $\vb{k}$ 
so that the kinetic part resembles the kinetic energy in a direct band gap system.
By redefining 
\begin{equation}
    \vb{k}_{1,2} = - \ii \hbar \partial_{\vb{r}_{1, 2}} + \frac{\vb{w}}{1 + \me / \mu}, \quad 
    \mu = \frac{\me \mh}{\me + \mh},
\end{equation}
and therefore 
\begin{equation}
    \begin{aligned}
        &\ke = \frac{\me}{M} \vb{P} + \frac{2 \mh \vb{w}}{M} - \vb{k}_1 - \vb{k}_2, \\ 
        &\khi{1,2} = \vb{k}_{1,2} - \frac{\mh \vb{w}}{M} + \frac{\mh}{M} \vb{P},
    \end{aligned}
\end{equation}
the total Hamiltonian now reads
\begin{equation}
    \begin{aligned}
        H &=  \frac{(\vb{P} - \vb{w})^2}{2 M} + \Eg \\ 
        &\ \  + \underbrace{
            \frac{\vb{k}_1^2 + \vb{k}_2^2}{2\mu} + \frac{\vb{k}_1 \cdot \vb{k}_2}{2\me}
            + V(\vb{r}_1 - \vb{r}_2) - V(\vb{r}_1) - V(\vb{r}_2)
        }_{H_{\text{internal}}}.
    \end{aligned}
    \label{eq:internal-ham}
\end{equation}
The redefinition of $\vb{k}_{1, 2}$ does not change the commutation relation 
between $\vb{r}_{1, 2}$ and $\vb{k}_{1, 2}$.
We find that the internal part of the Hamiltonian has the same form 
as the trion Hamiltonian in \cite{chang2021variationally}, 
which includes no indirect band gap.
This can also be seen as generalization of the indirect band gap exciton two-band model to the trion;
for the exciton case we have \cite{rustagi2018photoemission} 
\begin{equation}
    H = \frac{(\vb{Q} - \vb{w})^2}{2M} + \Eg + \frac{\vb{k}^2}{2\mu} - V(\vb{r}), 
\end{equation}
where 
\begin{equation}
    \begin{aligned}
        &M = \me + \me, \\
        &\ke = \frac{\me}{M} \vb{Q} + \vb{k} + \frac{\mh}{M} \vb{w}, \\
        &\kh = \frac{\mh}{M} \vb{Q} - \vb{k} - \frac{\mh}{M} \vb{w}.
    \end{aligned}
\end{equation}

After diagonalization of \eqref{eq:internal-ham}, we get a trion spectrum with the form of 
\begin{equation}
    E_{S \vb{P}} = \frac{(\vb{P} - \vb{w})^2}{2M} + \Eg + E^{\text{b}}_{S},
\end{equation}
where $E_S^{\text{b}}$ comes from diagonalization of $H_{\text{internal}}$.
The trion wave function coming from diagonalization of  $H_{\text{internal}}$ assumes the following form: 
\begin{equation}
    \psi_{S \vb{P}}(\vb{R}, \vb{r}_1, \vb{r}_2) = \ee^{\ii \vb{P} \cdot \vb{R}} \sum_{s_1, s_2} A_{s_1 s_2}^{S \vb{P}} \phi_{s_1}(\vb{r}_1) \phi_{s_2}(\vb{r}_2), 
\end{equation}
where $\phi_{s}(\vb{r})$ is an eigenfunction of the hydrogenic Hamiltonian $\frac{\vb{k}^2}{2\mu} - V(\vb{r})$, 
and $A_{s_1 s_2}^{S \vb{P}}$ reflects the mixing of the modes by interactive Hamiltonian $V(\vb{r}_1 - \vb{r}_2)$ and $\vb{k}_1 \cdot \vb{k}_2 / 2\me$.
Working in the momentum space we can also write 
\begin{equation}
    \psi_{S \vb{P}}(\vb{R}, \vb{p}_1, \vb{p}_2) = \ee^{\ii \vb{P} \cdot \vb{R}} \sum_{s_1, s_2} A_{s_1 s_2}^{S \vb{P}} \phi_{s_1}(\vb{p}_1) \phi_{s_2}(\vb{p}_2), 
\end{equation}
where $\phi_{s}(\vb{p})$ is the Fourier transform of $\phi_s(\vb{r})$.
The additional constraint required to pin down the relation between $\khi{1,2}$ and $\ke, \vb{P}$ is maximization of $\psi_{S \vb{P}}(\vb{R}, \vb{p}_1, \vb{p}_2)$.

\section{Two peak model}

Similarly for 
\begin{equation}
    \begin{aligned}
        H &= \frac{\ke^2}{2\me} + \Eg 
        + \frac{\khi{1}^2}{2\mh} + \frac{(\khi{2} - \vb{w})^2}{2\mh} \\
        &\ \ + V(\rhi{1} - \rhi{2}) - V(\re - \rhi{1}) - V(\re - \rhi{2}),
    \end{aligned}
\end{equation}
by defining 
\begin{equation}
    \vb{k}_1 = - \ii \hbar \partial_{\vb{r}_{1}} + \frac{\mh}{M} \vb{w}, \quad 
    \vb{k}_2 = - \ii \hbar \partial_{\vb{r}_{2}} - \frac{\me + \mh}{M} \vb{w},
\end{equation}
and hence 
\begin{equation}
    \begin{aligned}
        &\ke = \frac{\me}{M} (\vb{P} - \vb{w}) - \vb{k}_1 - \vb{k}_2, \\ 
        &\khi{1} = \vb{k}_1 - \frac{\mh}{M} \vb{w} + \frac{\mh}{M} \vb{P}, \\ 
        &\khi{2} = \vb{k}_2 + \frac{\me + \mh}{M} \vb{w} + \frac{\mh}{M} \vb{P},
    \end{aligned}
\end{equation}
we get exact the same Hamiltonian in \eqref{eq:internal-ham}.

When $\vb{k}_1 = \vb{k}_2$, we have 
\begin{equation}
    \khi{1} = \frac{1}{2} (\vb{P} - \vb{w} - \ke), \quad 
    \khi{2} = \frac{1}{2} (\vb{P} + \vb{w} - \ke), 
\end{equation}
and the dispersion relation is 
\begin{equation}
    \omega = E_{S\vb{P}} + \Eg + \frac{(\vb{P} - \vb{w} - \ke)^2}{4\mh}.
\end{equation}

In conclusion, the ARPES signature of any electron-hole-hole trion mode
has two major differences from that of an exciton mode with a comparable momentum:
the center and the curvature of the signature. 

\section{Two-electron case}

The shifts of the center and the curvature of ARPES signatures are even more clearly 
when the trion mode has two electrons and one hole.
A typical setting of such a trion mode can be described by the following two-band model, 
where the two electrons reside at different valleys:
\begin{equation}
    \begin{aligned}
    H &= \frac{\kei{1}^2}{2\me} + \frac{(\kei{2} - \vb{w})^2}{2\me} + \frac{\kh^2}{2\mh} + 2 \Eg \\
    &\quad + V(\rei{1} - \rei{2}) - V(\rh - \rei{1}) - V(\rh - \rei{2}).
    \end{aligned}
\end{equation}
The decomposition of the Hamiltonian into internal and external terms 
follows exactly the same procedure as in the two-peak model; 
we only need to swap the positions of electrons and holes. 
Therefore we find that under 
\begin{equation}
    \begin{aligned}
        \vb{r}_1 &= \rei{1} - \rh, \quad 
        \vb{r}_2 = \rei{2} - \rh, \\
        \vb{k}_1 &= - \ii \partial_{\vb{r}_1} + \frac{\me}{M} \vb{w}, \quad 
        \vb{k}_2 = - \ii \partial_{\vb{r}_2} - \frac{\me + \mh}{M} \vb{w}, 
    \end{aligned}
\end{equation}
or conversely 
\begin{equation}
    \begin{aligned}
        &\kh = \frac{\mh}{M} (\vb{P} - \vb{w}) - \vb{k}_1 - \vb{k}_2, \\
        &\kei{1} = \vb{k}_1 - \frac{\me}{M} \vb{w} + \frac{\me}{M} \vb{P}, \\
        &\kei{2} = \vb{k}_2 + \frac{\me + \mh}{M} \vb{w} + \frac{\me}{M} \vb{P} ,
    \end{aligned}
\end{equation}
the Hamiltonian of the two-electron-one-hole trion is reduced to 
% See src/two-electron.nb
\begin{equation}
    \begin{aligned}
        H &=  \frac{(\vb{P} - \vb{w})^2}{2 M} + 2 \Eg, \\ 
        &\ \  + \underbrace{
            \frac{\vb{k}_1^2}{2\mu} + \frac{\vb{k}_2^2}{2\mu} + \frac{\vb{k}_1 \cdot \vb{k}_2}{2\mh}
            + V(\vb{r}_1 - \vb{r}_2) - V(\vb{r}_1) - V(\vb{r}_2)
        }_{H_{\text{internal}}},
    \end{aligned}
\end{equation}
a form resembling \eqref{eq:internal-ham}, 
the only differences being the coefficient of the $\vb{k}_1 \cdot \vb{k}_2$ term
and the $2\Eg$ term due to the fact that there are two electrons now.
Now when one electron is driven out by the probe, 
it can be e1 or e2, so $\kp = \kei{1,2}$, 
momentum conservation blocks any block terms between the two valleys, 
so we can evaluate the ARPES signature contributed by the two valleys separately.
This drastically changes the dispersion relation of the ARPES signature, 
but the structure of trion wave function is intact.
The most remarkable difference between the ARPES signature of a negative trion 
and that of a positive trion
is the former has a vertical ``tail'': 
when $\kp = \kei{1}$ is chosen, $\vb{k}_2$ is allow to move freely, 
and it continuously adjusts the energy of the remaining of the trion after e1 is driven out, 
which means the ARPES signature will have a long, vertical component 
that does not coincide with any dispersion relations. % TODO: rework the wording

Therefore, the curvature of the ARPES signature can still be found by either postulating $\vb{k}_1 = \vb{k}_2$ or by postulating that one of the momenta vanishes.
When $\vb{k}_1 = \vb{k}_2$ and $\kp = \kei{1}$, we have 
\begin{equation}
    \kei{2} - \vb{w} = \kp, \quad \kh = \vb{P} - \vb{w} - 2 \kp, 
\end{equation}
and the dispersion relation is 
\begin{equation}
    \omega = \Eg + E_{\text{b}} + \frac{\vb{P}^2}{2M} - \frac{\kp^2}{2\me} - \frac{(2 \kp + \vb{w} - \vb{P})^2}{2\mh}, 
\end{equation}
while the effective mass is 

\section{Variational wave function}

The spin part is in a singlet state, 
and therefore the orbital part of the wave function is symmetric.

It should be noted that since $\vb{k}$ is confined on the two-dimensional place, 
so is $\vb{r}$, 
which is the canonical coordinate corresponding to the crystal momentum $\vb{k}$
instead of the free-space momentum, 
and therefore differs from the free-space position;
the wave function in terms of the free-space position does have $z$-directional distribution,
but the wave function in terms of $\vb{r}$ does not.
The spatial part of the wave function 
therefore is to be solved on the 2D plane.
On the other hand, electromagnetism in a two-dimensional system 
is still inherently three-dimensional, 
and the Coulomb force retains its inverse-square form,
leading to the $\sim 1/r$ form of $V(r)$.

The 3D version of the problem is solved in \cite{chandrasekhar1944some,hogaasen2010two};

\begin{equation}
    \Psi(\vb{r}_1, \vb{r}_2) = 
    \psi_{\text{1s,$a$}}(\vb{r}_1) 
    \psi_{\text{1s,$b$}}(\vb{r}_2)
    + \psi_{\text{1s,$b$}}(\vb{r}_1) 
    \psi_{\text{1s,$a$}}(\vb{r}_2)
\end{equation}

\begin{equation}
    \braket*{\text{1s,$a$}}{\text{1s,$b$}} = \frac{4ab}{(a+b)^2}.
\end{equation}
\begin{equation}
    \mel*{\text{1s,$a$}}{-\laplacian}{\text{1s,$b$}} = \frac{4}{(a+b)^2}.
\end{equation}

\begin{equation}
    \mel*{\text{1s, $a$}}{\frac{1}{r}}{\text{1s, $b$}} = \frac{4}{a+b}
\end{equation}

Serious problem:
\begin{equation}
    \int \dd[2]{\vb{r}_1} \int \dd[2]{\vb{r}_2} \ee^{- r_1 / a} \frac{1}{\abs*{\vb{r}_1 - \vb{r}_2}} \ee^{- r_2 / b}
\end{equation}
One option might be to numerically solve the interaction part, 
and use the analytic forms elsewhere. 

An immediate consequence of this fact is that
since at the ground state 
\begin{equation}
    \ke = \frac{\me}{M} \vb{P} + \frac{2 \mh}{M} \vb{w}.
\end{equation}

The same procedure can be applied to other configurations of peaks and valleys.

Variational wave function: TODO list 
\begin{enumerate}
    \item We should work in the real space, because the form of $V$ can be complicated.
    \item The kinetic term should be evaluated analytically.
    \item Normalization??? Why the factor is $1 / \sqrt{2}$?
    \item The structure of the three-particle wave function.
    \item Also see the paper Two-electron atoms, ions, and molecules http://users.df.uba.ar/dmitnik/estructura3/articulostrabajo/twoelectronatomsionsmolecules.pdf
    and https://articles.adsabs.harvard.edu/pdf/1944ApJ...100..176C
\end{enumerate}

\section{ARPES signatures}

When $\vb{k}_1$ is set to zero, we find 
\begin{equation}
    \khi{2} = \frac{\me + \mh}{M} \vb{P} + \frac{\mh}{M} \vb{w} - \ke,
\end{equation}

and when $\vb{k}_1 = \vb{k}_2$, we have 

It can be seen that the trion ARPES signature shows a significantly smaller variance with respect to the trion temperature
than the exciton ARPES signature does.

\bibliography{arpes,trion}


\end{document}