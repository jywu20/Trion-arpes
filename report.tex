\documentclass[t]{beamer}
\usepackage{physics}
\usepackage{amsmath}
\usepackage{tikz}
\usepackage{mathdots}
\usepackage{yhmath}
\usepackage{cancel}
\usepackage{color}
\usepackage{siunitx}
\usepackage{array}
\usepackage{multirow}
\usepackage[version=4]{mhchem}
\usepackage{amssymb}
\usepackage{textcomp, gensymb}
\usepackage{mathtools}
\usepackage{pifont}
\newcommand{\cmark}{\ding{51}}%
\newcommand{\xmark}{\ding{55}}%
\usepackage{fontawesome5}
\usepackage{tabularx}
\usepackage{extarrows}
\usepackage{booktabs}
\usetikzlibrary{fadings}
\usetikzlibrary{patterns}
\usetikzlibrary{shadows.blur}
\usetikzlibrary{shapes}
\usepackage[style=authoryear,backend=bibtex]{biblatex}
\addbibresource{gw.bib}
\renewcommand{\footnotesize}{\scriptsize}
\usepackage{listings}
\usepackage{hyperref}

\newcommand{\pair}[1]{\langle #1 \rangle}
\DeclareMathOperator{\ee}{e}
\DeclareMathOperator{\ii}{i}
\DeclareMathOperator{\sgn}{sgn}

\newcommand{\concept}[1]{\textbf{#1}}
\newcommand*{\abinitio}{\textit{ab initio}}
\newcommand{\shortcode}[1]{\texttt{#1}}
\newcommand*{\const}{\text{const}}

%region Theme 

\usetheme{madrid}

% Show section in foot
\makeatletter
\setbeamertemplate{footline}
{
  \leavevmode%
  \hbox{%
  \begin{beamercolorbox}[wd=.333333\paperwidth,ht=2.25ex,dp=1ex,center]{author in head/foot}%
    \usebeamerfont{author in head/foot}\insertauthor
  \end{beamercolorbox}%
  \begin{beamercolorbox}[wd=.333333\paperwidth,ht=2.25ex,dp=1ex,center]{title in head/foot}%
    \usebeamerfont{title in head/foot}\insertsection
  \end{beamercolorbox}%
  \begin{beamercolorbox}[wd=.333333\paperwidth,ht=2.25ex,dp=1ex,right]{date in head/foot}%
    \usebeamerfont{date in head/foot}\insertshortdate{}\hspace*{2em}
    \insertframenumber{} / \inserttotalframenumber\hspace*{2ex} 
  \end{beamercolorbox}}%
  \vskip0pt%
}
\makeatother

%endregion

%region  Disable unsupported commands in bookmark titles 
\pdfstringdefDisableCommands{%
  \def\\{}%
  \def\texttt#1{<#1>}%
  \def\mathbb#1{#1}%
}
\pdfstringdefDisableCommands{\def\eqref#1{(\ref{#1})}}

\makeatletter
\pdfstringdefDisableCommands{\let\HyPsd@CatcodeWarning\@gobble}
\makeatother

%endregion

%Remove navigation symbols
\setbeamertemplate{navigation symbols}{}
%Remove frame continuation numbering
\setbeamertemplate{frametitle continuation}{}


%Information to be included in the title page:
\title{Trion in time-resolved ARPES}
\author{Jinyuan Wu}

\begin{document}

\maketitle

\begin{frame}[allowframebreaks]
\frametitle{Theory of TR-ARPES}

{\color{gray} The most generic theory requires Keldysh formalism 
but let's spare ourselves the burden\dots}

\vspace{1cm}

\textbf{Ingredients of our model of ARPES}
\begin{itemize}
    \item Electric dipole interaction only
    \item Sudden approximation 
    \item \emph{Separation between pump and probe}:
    system \emph{not} driven when probed; 
    pump prepares an initial state and nothing more
    \item Fermi's golden rule in probing 
\end{itemize}

\framebreak

\textbf{Main result} Output intensity $I_{\vb{k}}(t) \propto \sum_c P^{c}_{\vb{k}}(t)$,
\[
    \scriptstyle
    P^c_{\vb{k}}(t) = \sum_{n, \vb{k}'}
    {\color{green} \rho_n }
    {\color{red} \abs*{M^{fc}_{\vb{k} \vb{k}'}}^2 }
    \int_{t_0}^t \dd{t_1} \int_{t_0}^t \dd{t_2}
    {\color{magenta} \ee^{-\ii (E_n - \omega ) (t_1 - t_2)}} 
    {\color{purple} \mel{\Psi_n(t_0)}{c^\dagger_{\vb{k}'} 
    U(t_2, t_1) c_{\vb{k}'}}{\Psi_n(t_0)}}
    {\color{teal} s(t_1) s(t_2)}.
\]

\begin{enumerate}
    \item $\color{green} \rho_n$: 
    distribution of the final state of pumping 
    (initial state of probing).
    \item Probe field is $\vb{E} = 
        s(t) 
        {\color{red} \vb{E}_0} 
        {\color{magenta} \ee^{- \ii \omega_0 t}} + \text{c.c.}$, and 
    the transition matrix is $\color{red}
    M^{fc}_{\vb{k} \vb{k}'} = \mel**{f \vb{k}}{- \vb{d} \cdot \vb{E}_0}{c \vb{k}'} $; 
    here $c$ is the band that is interacting with light, 
    $f$ is the out-going state.
    \item The $\color{magenta} \ee^{- \ii (E_n - \omega) (t_1 - t_2)}$ factor 
    gives half of energy conservation condition; 
    $\omega$ is driving frequency $\omega_0$ shifted by work function.
    \item $\color{purple} \mel**{\Psi_n}{\cdots}{\Psi_n}$: 
    electron Green function with excited state background $\ket*{\Psi_n}$; 
    it gives the structure of $\ket*{\Psi_n}$ in electron basis, 
    and the second half of energy conservation condition 
    (energy after one electron being kicked out);
    \item $\color{teal} s(t_1) s(t_2)$: shape of probe pulse; 
    broadening $\delta(E_n - \omega - E_{\text{after}})$
\end{enumerate}

\end{frame}

\begin{frame}[allowframebreaks]
\frametitle{Successful example: exciton}

\begin{itemize}
    \item Below we work with 2D material $\Rightarrow$ $\vb{k}^{\text{out}}_\parallel = \vb{k}'$; 
    we refer to $\vb{k}^{\text{out}}_\parallel$ as $\vb{k}$
    \item Only consider valence band top and conduction band valley:
    parabolic bands
\end{itemize}

\vspace{1cm}

\textbf{ARPES response of an exciton (besides valence band electron)}
\[
    \scriptstyle
    P(t) = \sum_{S, \vb{Q}} \rho_{S \vb{Q}} \abs{A_{\vb{k}'}^{S \vb{Q}}}^2 \abs{M^{fc}_{\vb{k} \vb{k}'}}^2 
    \int_{t_0}^t \dd{t_1} \int_{t_0}^t \dd{t_2}
    \ee^{-\ii (E_{S \vb{Q}} + \epsilon_{v \vb{k}' - \vb{Q}} - \omega ) (t_1 - t_2)} 
    s(t_1) s(t_2) .
\] 

\begin{itemize}
    \item Detection of $\abs*{A^{S \vb{Q}}_{\vb{k}}}^2$ 
    \item $\vb{Q} = 0$ $\Rightarrow$ dispersion relation $\omega = \epsilon_{v \vb{k}}$
    (energy conservation)
    \item $\vb{Q}$ obeys $\ee^{- \beta E_{\vb{Q}}}$ distribution:
    dispersion relation is $\omega = E_{\text{g}} + E_{\text{B}} + \frac{\vb{k}^2}{2 m_{\text{e}}} $
    (energy conservation; 
    $\abs*{A^{S \vb{Q}}_{\vb{k}}}^2$ reaches peak when $\vb{v}_{\text{e}} = \vb{v}_{\text{h}}$)
\end{itemize}


\framebreak

\begin{columns}

\begin{column}{0.45\textwidth}
    \centering
    \includegraphics[width=\textwidth]{images/exciton-Q-0.0.png}
    Single exciton, $\vb{Q} = \SI{0.0}{\angstrom^{-1}}$
\end{column}
    
\begin{column}{0.45\textwidth}
    \centering
    \includegraphics[width=\textwidth]{images/exciton-Q-0.5.png}
    Single exciton, $\vb{Q} = \SI{0.5}{\angstrom^{-1}}$ 
\end{column}

\end{columns}

\vspace{1cm}

\begin{itemize}
    \item The center of the signature is on  
    a replica of the conduction band  
    \item The shape of the signature is a replica of the valence band 
\end{itemize}

\framebreak

\begin{columns}

\begin{column}{0.45\textwidth}
    \centering
    \includegraphics[width=\textwidth]{images/exciton-beta-10.png}
    Thermalized exciton, $\beta = \SI{10}{eV}$
\end{column}
    
\begin{column}{0.45\textwidth}
    \centering
    \includegraphics[width=\textwidth]{images/exciton-beta-100.png}
    Thermalized exciton, $\beta = \SI{100}{eV}$
\end{column}

\end{columns}

\vspace{1cm}

\begin{itemize}
    \item The center of the signature is on 
    a replica of the conduction band  
    \item The shape of the signature is a replica of the valence band 
    \item When excitons are hot, 
    the dispersion relation is the conduction band; 
    when they are cold, the dispersion relation is the valence band
\end{itemize}

\end{frame}

\begin{frame}[allowframebreaks]
\frametitle{Trion: some theoretical issues}

In the current results:
\begin{itemize}
    \item Two holes, one electron (what likely happens in ARPES)
    \item \emph{Two holes on one band} -- to simply analysis 
    \item \emph{No scattering after the electron is kicked out} -- realistic or not?
\end{itemize}    

\textbf{Trion Hamiltonian} By some analysis:
\begin{equation}
    \begin{aligned}
        H_{\text{T}} &=
          \frac{\vb{k}_{\text{e}}^2}{2 m_{\text{e}}}
        + \frac{\vb{k}_{\text{h1}}^2}{2 m_{\text{h}}}
        + \frac{\vb{k}_{\text{h2}}^2}{2 m_{\text{h}}} 
        + V(\vb{r}_{\text{h1}} - \vb{r}_{\text{h2}}) 
        - V(\vb{r}_{\text{e}} - \vb{r}_{\text{h1}}) 
        - V(\vb{r}_{\text{e}} - \vb{r}_{\text{h2}})  \\ 
        &= \underbrace{
        H_\text{X}(\vb{r}_1) + H_\text{X}(\vb{r}_2) 
        - \frac{\hbar^2 \laplacian}{2 \mu} + V(\vb{r}_1 - \vb{r}_2) - E_{\text{g}}
        }_{\text{causing $E_{\text{B}}$}}
        + \frac{\vb{P}_{\text{T}}^2}{2 m_{\text{T}}} + E_{\text{g}},
    \end{aligned}
\end{equation}
where 
\begin{equation}
    \scriptstyle
    m_\text{T} = \underbrace{2 m_{\text{h}}}_{>0} + m_{\text{e}}, \quad 
    \vb{k}_\text{h1} = \vb{k}_1 + \frac{m_{\text{h}}}{m_{\text{T}}} \vb{P}_{\text{T}}, \quad 
    \vb{k}_\text{h2} = \vb{k}_2 + \frac{m_{\text{h}}}{m_{\text{T}}} \vb{P}_{\text{T}}, \quad 
    \vb{k}_{\text{e}} = \frac{m_{\text{e}}}{m_{\text{T}}} \vb{P}_{\text{T}} - \vb{k}_1 - \vb{k}_2.
\end{equation}

\framebreak

That's to say: a trion, when the two holes of it are of the same species, 
can be seen as the bound state of two excitons.

\textbf{Trion wave function} Thus in this specific case: 
\begin{equation}
    \phi_{\text{T}}(\vb{r}_{\text{e}}, \vb{r}_{\text{h1}}, \vb{r}_{\text{h2}})
    = \sum_{S_1, S_2} \phi_{\text{X}}^S(\vb{r}_1) \phi_{\text{X}}^S(\vb{r}_2) \ee^{\ii \vb{P}_{\text{T}} \cdot \vb{R}},
\end{equation}
where $\vb{r}_{1, 2}$ are defined above.

\vspace{1cm}

\textbf{In the tentative plots below} We ignore spin and choose $S_1 = S_2 = \text{1s}$.
The ARPES response (here $\epsilon_v$ are \emph{negative}):
\begin{equation}
    \begin{aligned}
        \scriptstyle
        P_{\vb{k}}(t) &= \sum_{\vb{P}_{\text{T}}} \rho_{\vb{P}_{\text{T}}} \sum_{\vb{k}_\text{h1}, \vb{k}_{\text{h2}}} 
        \sum_{S_1, S_2} \abs*{M^{fc}_{\vb{k} \vb{k}_{\text{e}}}}^2 \abs*{A_{S_1 S_2}}^2
        \abs*{\phi_{S_1}(\vb{k}_{\text{1}})}^2 
        \abs*{\phi_{S_2}(\vb{k}_{\text{2}})}^2 \\
        &\quad \times \int_{t_0}^t \dd{t_1} \int_{t_0}^t \dd{t_2}
        \ee^{-\ii (E_{S \vb{P}_{\text{T}}} + \epsilon_{v \vb{k}_{\text{h1}}} + \epsilon_{v \vb{k}_{\text{h2}}} - \omega ) (t_1 - t_2)} 
        s(t_1) s(t_2)
    \end{aligned}
\end{equation}

\end{frame}

\begin{frame}
\frametitle{Trion: tentative results}

\begin{columns}

\begin{column}{0.45\textwidth}
    
\includegraphics[width=\textwidth]{images/trion-P-0.0.png}
Single trion, $\vb{P} = \SI{0.0}{\angstrom^{-1}}$

\end{column}

\begin{column}{0.45\textwidth}
    
\includegraphics[width=\textwidth]{images/trion-P-0.5.png}
Single trion, $\vb{P} = \SI{0.5}{\angstrom^{-1}}$

\end{column}

\end{columns}    

\vspace{1cm}

\begin{itemize}
    \only<1>{\item The center of the signature is still sticked to a replica of the conduction band
    (moved downwards by $E_{\text{B}}$):
    the maximum of a signature is reached when $\vb{k}_1 = \vb{k}_2 = 0$
    where $E_{\text{T}} = \epsilon_v - E_{\text{B}}$}
    \only<2>{\item The analytic form of the shape of a single-trion signature is still not clear; 
    I tried to set one of $\vb{k}_{1, 2}$ to zero, 
    or $\vb{k}_1 = \vb{k}_2$; 
    both of them seem to work, but not that well; 
    the dispersion relation given by the first (but not the second) condition 
    is also a replica of $\epsilon_\text{v}$.} 
\end{itemize}

\end{frame}

\begin{frame}
\frametitle{Trion: tentative results}

\begin{columns}

\only<1>{\begin{column}{0.45\textwidth}
\centering
    \includegraphics[width=\textwidth]{images/exciton-Q-0.0.png}
    Exciton, $\vb{Q} = \SI{0.0}{\angstrom^{-1}}$
\end{column}

\begin{column}{0.45\textwidth}
    \centering
    \includegraphics[width=\textwidth]{images/trion-P-0.0.png}
    Trion, $\vb{P} = \SI{0.0}{\angstrom^{-1}}$
\end{column}}

\only<2>{\begin{column}{0.45\textwidth}
\centering
    \includegraphics[width=\textwidth]{images/exciton-Q-0.5.png}
    Exciton, $\vb{Q} = \SI{0.5}{\angstrom^{-1}}$
\end{column}

\begin{column}{0.45\textwidth}
    \centering
    \includegraphics[width=\textwidth]{images/trion-P-0.5.png}
    Trion, $\vb{P} = \SI{0.5}{\angstrom^{-1}}$
\end{column}}

\end{columns}

\end{frame}

\begin{frame}
\frametitle{Conclusion}

\begin{itemize}
    \item Trion signature on ARPES spectrum looks similar to that of exciton 
    \item The finite-$T$ induced ``upper'' dispersion of the two is the same 
    \item But the shape and spread of a single trion mode is still different from   
    those of an exciton mode, 
    even with close $E_{\text{B}}$
\end{itemize}

\end{frame}

\end{document}