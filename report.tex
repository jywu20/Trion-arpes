\documentclass[t,aspectratio=169]{beamer}
\usepackage{physics}
\usepackage{amsmath}
\usepackage{tikz}
\usepackage{mathdots}
\usepackage{yhmath}
\usepackage{cancel}
\usepackage{color}
\usepackage{siunitx}
\usepackage{array}
\usepackage{multirow}
\usepackage[version=4]{mhchem}
\usepackage{amssymb}
\usepackage{textcomp, gensymb}
\usepackage{mathtools}
\usepackage{pifont}
\newcommand{\cmark}{\ding{51}}%
\newcommand{\xmark}{\ding{55}}%
\usepackage{fontawesome5}
\usepackage{tabularx}
\usepackage{extarrows}
\usepackage{booktabs}
\usetikzlibrary{fadings}
\usetikzlibrary{patterns}
\usetikzlibrary{shadows.blur}
\usetikzlibrary{shapes}
\usepackage[style=authoryear,backend=bibtex]{biblatex}
\addbibresource{gw.bib}
\renewcommand{\footnotesize}{\scriptsize}
\usepackage{listings}
\usepackage{hyperref}

\newcommand{\pair}[1]{\langle #1 \rangle}
\DeclareMathOperator{\ee}{e}
\DeclareMathOperator{\ii}{i}
\DeclareMathOperator{\sgn}{sgn}

\newcommand{\concept}[1]{\textbf{#1}}
\newcommand*{\abinitio}{\textit{ab initio}}
\newcommand{\shortcode}[1]{\texttt{#1}}
\newcommand*{\const}{\text{const}}

%region Theme 

\usetheme{metropolis}

% Show section in foot
\makeatletter
\setbeamertemplate{footline}
{
  \leavevmode%
  \hbox{%
  \begin{beamercolorbox}[wd=.333333\paperwidth,ht=2.25ex,dp=1ex,center]{author in head/foot}%
    \usebeamerfont{author in head/foot}\insertauthor
  \end{beamercolorbox}%
  \begin{beamercolorbox}[wd=.333333\paperwidth,ht=2.25ex,dp=1ex,center]{title in head/foot}%
    \usebeamerfont{title in head/foot}\insertsection
  \end{beamercolorbox}%
  \begin{beamercolorbox}[wd=.333333\paperwidth,ht=2.25ex,dp=1ex,right]{date in head/foot}%
    \usebeamerfont{date in head/foot}\insertshortdate{}\hspace*{2em}
    \insertframenumber{} / \inserttotalframenumber\hspace*{2ex} 
  \end{beamercolorbox}}%
  \vskip0pt%
}
\makeatother

%endregion

%region  Disable unsupported commands in bookmark titles 
\pdfstringdefDisableCommands{%
  \def\\{}%
  \def\texttt#1{<#1>}%
  \def\mathbb#1{#1}%
}
\pdfstringdefDisableCommands{\def\eqref#1{(\ref{#1})}}

\makeatletter
\pdfstringdefDisableCommands{\let\HyPsd@CatcodeWarning\@gobble}
\makeatother

%endregion

%Remove navigation symbols
\setbeamertemplate{navigation symbols}{}
%Remove frame continuation numbering
\setbeamertemplate{frametitle continuation}{}

% Macros for two-band model
\newcommand*{\ke}{\vb{k}_{\text{e}}}
\newcommand*{\khi}[1]{\vb{k}_{\text{h#1}}}
\newcommand*{\kh}{\vb{k}_{\text{h}}}
\newcommand*{\re}{\vb{r}_{\text{e}}}
\newcommand*{\rhi}[1]{\vb{r}_{\text{h#1}}}
\newcommand*{\me}{m_{\text{e}}}
\newcommand*{\mh}{m_{\text{h}}}
\newcommand*{\Eg}{E_{\text{g}}}

%Information to be included in the title page:
\title{Trion in time-resolved ARPES}
\author{Jinyuan Wu}

\begin{document}

\maketitle

\begin{frame}[allowframebreaks]
\frametitle{Theory of TR-ARPES}

{\color{gray} The most generic theory requires Keldysh formalism 
but let's spare ourselves the burden\dots}

\vspace{1cm}

\textbf{Ingredients of our model of ARPES}
\begin{itemize}
    \item Electric dipole interaction only
    \item Sudden approximation 
    \item \emph{Separation between pump and probe}:
    system \emph{not} driven when probed; 
    pump prepares an initial state and nothing more
    \item Fermi's golden rule in probing 
\end{itemize}

\framebreak

\textbf{Main result} Output intensity $I_{\vb{k}}(t) \propto \sum_c P^{c}_{\vb{k}}(t)$,
\scriptsize
\[ 
    P^c_{\vb{k}}(t) = \sum_{n, \vb{k}'}
    {\color{green} \rho_n }
    {\color{red} \abs*{M^{fc}_{\vb{k} \vb{k}'}}^2 }
    \int_{t_0}^t \dd{t_1} \int_{t_0}^t \dd{t_2}
    {\color{magenta} \ee^{-\ii (E_n - \omega ) (t_1 - t_2)}} 
    {\color{purple} \mel{\Psi_n(t_0)}{c^\dagger_{\vb{k}'} 
    U(t_2, t_1) c_{\vb{k}'}}{\Psi_n(t_0)}}
    {\color{teal} s(t_1) s(t_2)}.
\]
\normalsize

\vspace{-0.5cm}

\small

\begin{enumerate}
    \item $\color{green} \rho_n$: 
    distribution of the final state of pumping 
    (initial state of probing).
    \item Probe field is $\vb{E} = 
        s(t) 
        {\color{red} \vb{E}_0} 
        {\color{magenta} \ee^{- \ii \omega_0 t}} + \text{c.c.}$, and 
    the transition matrix is $\color{red}
    M^{fc}_{\vb{k} \vb{k}'} = \mel**{f \vb{k}}{- \vb{d} \cdot \vb{E}_0}{c \vb{k}'} $; 
    $c$ is the probed electron, 
    $f$ is the out-going state.
    \item The $\color{magenta} \ee^{- \ii (E_n - \omega) (t_1 - t_2)}$ factor 
    gives half of energy conservation condition; 
    $\omega$ is driving frequency $\omega_0$ shifted by work function.
    \item $\color{purple} \mel**{\Psi_n}{\cdots}{\Psi_n}$: 
    electron Green function with excited state background $\ket*{\Psi_n}$; 
    it gives the structure of $\ket*{\Psi_n}$ in electron basis, 
    and the second half of energy conservation condition 
    (energy after one electron being kicked out);
    \item $\color{teal} s(t_1) s(t_2)$: shape of probe pulse; 
    broadening $\delta(E_n - \omega - E_{\text{after}})$
\end{enumerate}

\normalsize

\end{frame}

\begin{frame}[allowframebreaks]
\frametitle{Successful example: exciton}

\begin{itemize}
    \item Below we work with 2D material $\Rightarrow$ $\vb{k}^{\text{out}}_\parallel = \vb{k}'$; 
    we refer to $\vb{k}^{\text{out}}_\parallel$ as $\vb{k}$
    \item Only consider valence band top and conduction band valley:
    parabolic bands
\end{itemize}

\vspace{1cm}

\textbf{ARPES response of an exciton (besides valence band electron)}
\scriptsize
\[
    P(t) = \sum_{S, \vb{Q}} \rho_{S \vb{Q}} \abs{A_{\vb{k}'}^{S \vb{Q}}}^2 \abs{M^{fc}_{\vb{k} \vb{k}'}}^2 
    \int_{t_0}^t \dd{t_1} \int_{t_0}^t \dd{t_2}
    \ee^{-\ii (E_{S \vb{Q}} + \epsilon_{v \vb{k}' - \vb{Q}} - \omega ) (t_1 - t_2)} 
    s(t_1) s(t_2) .
\] 
\normalsize

\begin{itemize}
    \item Detection of $\abs*{A^{S \vb{Q}}_{\vb{k}}}^2$ 
    \item $\vb{Q} = 0$ $\Rightarrow$ dispersion relation $\omega = \epsilon_{v \vb{k}}$
    (energy conservation)
    \item $\vb{Q}$ obeys $\ee^{- \beta E_{\vb{Q}}}$ distribution:
    dispersion relation is $\omega = E_{\text{g}} + E_{\text{B}} + \frac{\vb{k}^2}{2 m_{\text{e}}} $
    (energy conservation; 
    $\abs*{A^{S \vb{Q}}_{\vb{k}}}^2$ reaches peak when $\vb{v}_{\text{e}} = \vb{v}_{\text{h}}$)
\end{itemize}


\framebreak

\begin{columns}

\begin{column}{0.45\textwidth}
    \centering
    \includegraphics[width=\textwidth]{images/exciton-Q-0.0.png}
    Single exciton, $\vb{Q} = \SI{0.0}{\angstrom^{-1}}$
\end{column}
    
\begin{column}{0.45\textwidth}
    \centering
    \includegraphics[width=\textwidth]{images/exciton-Q-0.5.png}
    Single exciton, $\vb{Q} = \SI{0.5}{\angstrom^{-1}}$ 
\end{column}

\end{columns}


\begin{itemize}
    \item The center of the signature is on  
    a replica of the conduction band  
    \item The shape of the signature is a replica of the valence band 
\end{itemize}

\framebreak

\begin{columns}

\begin{column}{0.4\textwidth}
    \centering
    \includegraphics[width=\textwidth]{images/exciton-beta-10.png}
    Thermalized exciton, $\beta = \SI{10}{eV}$
\end{column}
    
\begin{column}{0.4\textwidth}
    \centering
    \includegraphics[width=\textwidth]{images/exciton-beta-100.png}
    Thermalized exciton, $\beta = \SI{100}{eV}$
\end{column}

\end{columns}


\begin{itemize}
    \item The center of the signature is on 
    a replica of the conduction band  
    \item The shape of the signature is a replica of the valence band 
    \item When excitons are hot, 
    the dispersion relation is the conduction band; 
    when they are cold, the dispersion relation is the valence band
\end{itemize}

\end{frame}

\begin{frame}[allowframebreaks]
\frametitle{Trion in two-band model}

Configuration of trion structure
\begin{itemize}
    \item Two holes, one electron (what likely happens in ARPES)
    \item \emph{Two holes on one band} -- to simplify analysis 
    \item \emph{No hole scattering after the electron is kicked out} -- realistic or not?
\end{itemize}    

\begin{equation}
    H =
          \frac{(\vb{k}_{\text{e}} - \vb{w})^2}{2 m_{\text{e}}}
    + \Eg
    + \frac{\vb{k}_{\text{h1}}^2}{2 m_{\text{h}}}
    + \frac{\vb{k}_{\text{h2}}^2}{2 m_{\text{h}}} 
    + V(\vb{r}_{\text{h1}} - \vb{r}_{\text{h2}}) 
    - V(\vb{r}_{\text{e}} - \vb{r}_{\text{h1}}) 
    - V(\vb{r}_{\text{e}} - \vb{r}_{\text{h2}}) .
\end{equation}

\textbf{Redefining coordinates} What we want: $\vb{k}_{\text{e}, \text{h1}, \text{h2}}$ $\to$ $\vb{P}, \vb{k}_1, \vb{k}_2$, following these constraints:
\[
    \vb{k}_{1, 2} = - \ii \hbar \partial_{\vb{r}_{1,2}} = \const, \quad \vb{r}_{1,2} = \vb{r}_{\text{h1,2}} - \vb{r}_{\text{e}}, \quad 
    \vb{R} = \ii \partial_{\vb{P}} = \frac{m_{\text{h}} \vb{r}_{\text{h1}} + m_{\text{h}} \vb{r}_{\text{h2}} + m_{\text{e}} \vb{r}_{\text{e}}}{2m_{\text{h}} + m_{\text{e}}}.
\]

It can be verified that with 
\begin{equation}
    \vb{P} = \ke + \khi{1} + \khi{2}, \quad 
    \vb{k}_{1,2} = - \ii \hbar \partial_{\vb{r}_{1, 2}} + \frac{\mh}{2\mh + \me} \vb{w},
\end{equation}
\begin{equation}
    M = 2\mh + \me , \quad \mu = \frac{\me \mh}{\me + \mh} , \quad 
    H_\text{X} = - \frac{\laplacian}{2\mu} + V(\vb{r}),
\end{equation}
\begin{equation}
    H = \underbrace{
    H_\text{X}(\vb{r}_1) + H_\text{X}(\vb{r}_2) 
    - \frac{\hbar^2 \laplacian}{2 \me} + V(\vb{r}_1 - \vb{r}_2) 
    }_{\text{causing $E_{\text{B}}$}}
    + \frac{\vb{P}^2}{2 m_{\text{T}}} + E_{\text{g}}
\end{equation}

\framebreak

That's to say: a trion, when the two holes of it are of the same species, 
can be seen as the bound state of two excitons.

\textbf{Trion wave function} Thus in this specific case: 
\begin{equation}
    \phi_{\text{T}}(\vb{r}_{\text{e}}, \vb{r}_{\text{h1}}, \vb{r}_{\text{h2}})
    = \sum_{S_1, S_2} \phi_{\text{X}}^S(\vb{r}_1) \phi_{\text{X}}^S(\vb{r}_2) \ee^{\ii \vb{P} \cdot \vb{R}},
\end{equation}
where $\vb{r}_{1, 2}$ are defined above.

The below ansatz is frequently used:
\begin{equation}
    \phi_\text{T}(\vb{r}_1, \vb{r}_2) \propto 
    \phi_{\text{1s}}(\vb{r}_1; a) \phi_{\text{1s}}(\vb{r}_2; b) +  \phi_{\text{1s}}(\vb{r}_1; b) \phi_{\text{1s}}(\vb{r}_2; a)
\end{equation}
where $a, b$ are exciton radii.

\vspace{1cm}

\textbf{In the tentative plots below} We ignore spin and choose $S_1 = S_2 = \text{1s}$.
The ARPES response (here $\epsilon_v$ are \emph{negative}):
\begin{equation}
    \begin{aligned}
        \scriptstyle
        P_{\vb{k}}(t) &= \sum_{\vb{P}} \rho_{\vb{P}} \sum_{\vb{k}_\text{h1}, \vb{k}_{\text{h2}}} 
        \sum_{S_1, S_2} \abs*{M^{fc}_{\vb{k} \vb{k}_{\text{e}}}}^2 \abs*{A_{S_1 S_2}}^2
        \abs*{\phi_{S_1}(\vb{k}_{\text{1}})}^2 
        \abs*{\phi_{S_2}(\vb{k}_{\text{2}})}^2 \\
        &\quad \times \int_{t_0}^t \dd{t_1} \int_{t_0}^t \dd{t_2}
        \ee^{-\ii (E_{S \vb{P}} + \epsilon_{v \vb{k}_{\text{h1}}} + \epsilon_{v \vb{k}_{\text{h2}}} - \omega ) (t_1 - t_2)} 
        s(t_1) s(t_2)
    \end{aligned}
\end{equation}

Signature maximum: $\vb{k}_1 = 0$ or $\vb{k}_2 = 0$, or $\vb{k}_1 = \vb{k}_2$?

\end{frame}

\begin{frame}
\frametitle{Trion and exciton: $\vb{Q} = \vb{w}$}

\begin{center}
    \includegraphics[width=0.4\textwidth]{images/exciton-Q-0.7-w-0.7.png}
    \includegraphics[width=0.4\textwidth]{images/trion-P-0.7-w-0.7.png}
\end{center}    

Main difference: curvature. In trion the signature partially has the shape of $2\epsilon_{\text{v}, \vb{k}/2}$

\end{frame}

\begin{frame}
\frametitle{Exciton and trion: finite $\vb{Q} - \vb{w}$}

\begin{center}
    \includegraphics[width=0.4\textwidth]{images/exciton-Q-1.2-w-0.7.png}
    \includegraphics[width=0.4\textwidth]{images/trion-P-1.2-w-0.7.png}
\end{center}

\begin{itemize}
    \item The signature curvature difference is more obvious.
    \item The positions of the signatures are different! 
    \begin{equation*}
        \vb{k}_{\text{exciton, max}} = \frac{\me \vb{Q} + \mh \vb{w}}{\me + \mh}, \quad 
        \vb{k}_{\text{trion, max}} = \frac{\me}{2\mh + \me} \vb{P} + \frac{2\mh}{2\mh + \me} \vb{w}.
    \end{equation*}
\end{itemize}

\end{frame}

\begin{frame}
\frametitle{Exciton and trion: thermalized, $\beta = 10$}

\begin{center}
    \includegraphics[width=0.4\textwidth]{images/exciton-β-10-w-0.7.png}
    \includegraphics[width=0.4\textwidth]{images/trion-β-10-w-0.7.png}
\end{center}    

The trion calculation is not well-conserved: the sampling grid of $\vb{P}$ is too small; 
I don't expect to see any large difference because the shape of the two signatures should both follow $\epsilon_{\text{c}}$.

\end{frame}

\begin{frame}
\frametitle{Conclusion}

\begin{itemize}
    \item Trion signature on ARPES spectrum looks similar to that of exciton 
    \item The finite-$T$ induced ``upper'' dispersion of the two is the same 
    \item The trion and exciton binding energies are different 
    \item With finite momenta, exciton and trion signatures are different in shape and position. 
\end{itemize}

\end{frame}

\end{document}